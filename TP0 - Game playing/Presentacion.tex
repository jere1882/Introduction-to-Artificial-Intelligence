%%%%%%%%%%%%%%%%%%%%%%%%%%%%%%%%%%%%%%%%%
% Beamer Presentation
% LaTeX Template
% Version 1.0 (10/11/12)
%
% This template has been downloaded from:
% http://www.LaTeXTemplates.com
%
% License:
% CC BY-NC-SA 3.0 (http://creativecommons.org/licenses/by-nc-sa/3.0/)
%
%%%%%%%%%%%%%%%%%%%%%%%%%%%%%%%%%%%%%%%%%

%----------------------------------------------------------------------------------------
%	PACKAGES AND THEMES
%----------------------------------------------------------------------------------------

\documentclass{beamer}

\mode<presentation> {

% The Beamer class comes with a number of default slide themes
% which change the colors and layouts of slides. Below this is a list
% of all the themes, uncomment each in turn to see what they look like.

%\usetheme{default}
%\usetheme{AnnArbor}
%\usetheme{Antibes}
%\usetheme{Bergen}
%\usetheme{Berkeley}
%\usetheme{Berlin}
%\usetheme{Boadilla}
%\usetheme{CambridgeUS}
%\usetheme{Copenhagen}
%\usetheme{Darmstadt}
%\usetheme{Dresden}
%\usetheme{Frankfurt}
%\usetheme{Goettingen}
%\usetheme{Hannover}
%\usetheme{Ilmenau}
%\usetheme{JuanLesPins}
%\usetheme{Luebeck}
\usetheme{Madrid}
%\usetheme{Malmoe}
%\usetheme{Marburg}
%\usetheme{Montpellier}
%\usetheme{PaloAlto}
%\usetheme{Pittsburgh}
%\usetheme{Rochester}
%\usetheme{Singapore}
%\usetheme{Szeged}
%\usetheme{Warsaw}

% As well as themes, the Beamer class has a number of color themes
% for any slide theme. Uncomment each of these in turn to see how it
% changes the colors of your current slide theme.

%\usecolortheme{albatross}
%\usecolortheme{beaver}
%\usecolortheme{beetle}
%\usecolortheme{crane}
%\usecolortheme{dolphin}
%\usecolortheme{dove}
%\usecolortheme{fly}
%\usecolortheme{lily}
%\usecolortheme{orchid}
%\usecolortheme{rose}
%\usecolortheme{seagull}
%\usecolortheme{seahorse}
%\usecolortheme{whale}
%\usecolortheme{wolverine}

%\setbeamertemplate{footline} % To remove the footer line in all slides uncomment this line
%\setbeamertemplate{footline}[page number] % To replace the footer line in all slides with a simple slide count uncomment this line

%\setbeamertemplate{navigation symbols}{} % To remove the navigation symbols from the bottom of all slides uncomment this line
}

\usepackage{graphicx} % Allows including images
\usepackage{booktabs} % Allows the use of \toprule, \midrule and \bottomrule in tables
\usepackage[spanish]{babel} 
\usepackage[utf8]{inputenc} 
\usepackage{multirow} % para las tablas
\usepackage[flushleft]{threeparttable}
\usepackage{verbatim}
%----------------------------------------------------------------------------------------
%	TITLE PAGE
%----------------------------------------------------------------------------------------

\title[]{General Game Playing} % The short title appears at the bottom of every slide, the full title is only on the title page

\subtitle{Representing and Reasoning About the Rules of General Games With Imperfect Information\footnote{Escrito por Stephan Schieffel y Michael Thielscher}}
\author[]{\scriptsize Aldana Ramirez, Jeremías Rodríguez, Sebastián Meli} % Your name
\institute[UNR] % Your institution as it will appear on the bottom of every slide, may be shorthand to save space
{
\small Introducción a la Inteligencia Artificial\\
Universidad Nacional de Rosario % Your institution for the title page
}
\date{\today} % Date, can be changed to a custom date

\begin{document}

\begin{frame}
\titlepage % Print the title page as the first slide
\end{frame}


%----------------------------------------------------------------------------------------
%	PRESENTATION SLIDES
%----------------------------------------------------------------------------------------

%------------------------------------------------


\begin{frame}
\frametitle{Conceptos de IA involucrados}
\begin{itemize}
\item Knowleadge
\item Learning
\item Reasoning
\item Planning
\item Decision Making
\item Logic Programming
\item Situation Calculus
\end{itemize}
\end{frame}


%------------------------------------------------

\begin{frame}
\frametitle{General Game Playing}
\begin{block}{General Game Playing (GGP)}
Es un sistema que aprende a jugar juegos previamente desconocidos, sin intervención humana, solo a partir de sus reglas.
\end{block}

\begin{block}{¿Cómo representamos las reglas?}
Se ha desarrollado un lenguaje de alto nivel llamado Game Description Language (GDL) que permite describir cualquier juego con un número finito de jugadores y movimientos legales en cada estado. \\
Los movimientos son determísticos y completamente observables, y todas las reglas del juego son conocidas por todos los jugadores (información completa). \\
Las reglas son fáciles de entender y mantener, y la semántica es precisa y totalmente procesable por una máquina.\\
Ejemplos: Ajedrez, Go.
\end{block}


\end{frame}

%------------------------------------------------
\begin{frame}
\frametitle{Game Description Language with Imperfect Information}

Este paper propone una extensión al lenguaje GDL para incluir aleatoriedad y conocimiento incompleto del estado del juego, llamado Game Description Language with Imperfect Information (GDL-II). \\
Permite describir juegos con:

\begin{itemize}
\item {Un número finito de jugadores que conocen todas las reglas de juego pero pueden tener información {\bf incompleta o asimétrica}.}
\item {Un número finito de movimientos legales que pueden ser \bf{no determinísticos}.}
\end{itemize}

Ejemplos: Poker, Bridge, Backgammon 

\end{frame}
%------------------------------------------------

\begin{frame}
\frametitle{GDL-II - Sintaxis}

La sintaxis de GDL-II está basada en la sintaxis estándar de los programas lógicos (incluyendo negación).\\
Una descripción de un juego es un conjunto finito de sentencias lógicas usando los predicados de la tabla 1, donde se imponen ciertas restricciones. 

\begin{center}
\begin{table}\scriptsize
\begin{tabular}{|l|l|}
\hline
role(R) & R es un jugador. \\
init(F) & F vale en la posición inicial. \\ \hline
true(F) & F vale en la posición actual. \\ 
legal(R,M) & El jugador R puede hacer el movimiento M en la posición actual. \\ 
does(R,M) & El jugador R hace el movimiento M. \\ 
next(F) & F vale en la posición siguiente. \\ \hline
terminal & La posición actual es terminal. \\
goal(R,V) & El jugador R gana V puntos en la posición actual. \\ \hline \hline
sees(R,P) & El jugador R percive P en la siguiente posición. \\
random &  El jugador aleatorio.\footnote{Notar que no necesariamente es un jugador convencional.} \\ \hline
\end{tabular}
\label{tabla:sencilla}
\end{table}
\end{center}

\vspace{-20pt}

\begin{center}
\scriptsize Tabla 1: GDL-II keywords
\end{center}

\end{frame}


\begin{frame}[fragile] % Need to use the fragile option when verbatim is used in the slide
\frametitle{Ejemplo: Krieg-Tic Tac Toe\footnote{Variante del clásico Tic Tac Toe donde los jugadores no pueden percibir los movimientos de su oponente}}
\begin{block}{Especificando las reglas en GDL-II}
\scriptsize \begin{verbatim}
1  role(xplayer).
2  role(oplayer).
3
4  init(control(xplayer)).
5  init(cell(1,1,b)). init(cell(1,2,b)). init(cell(1,3,b)).
6  init(cell(2,1,b)). init(cell(2,2,b)). init(cell(2,3,b)).
7  init(cell(3,1,b)). init(cell(3,2,b)). init(cell(3,3,b)).
8
9  legal(xplayer,mark(M,N)) :- true(control(xplayer)), true(cell(M,N,Z)),
10                             distinct(Z,x), not true(tried(xplayer,M,N)).
11 legal(oplayer,mark(M,N)) :- true(control(oplayer)), true(cell(M,N,Z)),
12                             distinct(Z,o), not true(tried(oplayer,M,N)).
13 legal(xplayer,noop)      :- true(control(oplayer)).
14 legal(oplayer,noop)      :- true(control(xplayer)).
\end{verbatim} 
\end{block}
\small {
Líneas 1-7  : Se establecen los dos roles y las posiciones iniciales.\\
Líneas 9-14 : Se especifican los movimientos }

\end{frame}


\begin{frame}[fragile] % Need to use the fragile option when verbatim is used in the slide
\frametitle{Ejemplo (continuación)}
\begin{block} {  }
\scriptsize \begin{verbatim}
16 validmove              :- does(R,mark(M,N)), true(cell(M,N,b)).
17
18 next(F)                :- not validmove, true(F).
19 next(tried(R,M,N))     :- not validmove, does(R,mark(M,N)).
20
21 next(cell(M,N,x))      :- validmove, does(xplayer,mark(M,N)).
22 next(cell(M,N,o))      :- validmove, does(oplayer,mark(M,N)).
23 next(cell(M,N,Z))      :- validmove, true(cell(M,N,Z)),
24                                      does(R,mark(I,J)), distinct(M,I).
25 next(cell(M,N,Z))      :- validmove, true(cell(M,N,Z)),
26                                      does(R,mark(I,J)), distinct(N,J).
27 next(control(oplayer)) :- validmove, true(control(xplayer)).
28 next(control(xplayer)) :- validmove, true(control(oplayer)).
29 next(tried(R,M,N))     :- validmove, true(tried(R,M,N)).
30
31 sees(R, yourmove)      :- not validmove, true(control(R)).
32 sees(xplayer,yourmove) :- validmove, true(control(oplayer)).
33 sees(oplayer,yourmove) :- validmove, true(control(xplayer)).
\end{verbatim} 
\end{block}
\small{
Líneas 18-29 : Se especifica cómo se actualizarán las posiciones.  \\
Líneas 31-33 : Se notificará a cada jugador cuando sea su turno. \\
Omitimos las condiciones de terminación y puntuación.
}
\end{frame}


\begin{frame}
\frametitle{GDL-II - Semántica}

Sea G una descripción válida para un juego en GDL-II.\\ 
Se obtiene un único modelo para G utilizando el concepto de {\bf modelo estable}.\footnote{El concepto de modelo estable fue introducido por M. Gelfond y V. Lifschitz en su paper {\it The Stable Model Semantics for Logic Programming} para proveer una semántica para programas lógicos con negación.}\\
\vspace{10pt}
La semántica de G está dada por un sistema de transición de estados, construído a partir del modelo estable.

\end{frame}

%------------------------------------------------

\begin{frame}
\frametitle{Razonando sobre GDL-II}

\begin{block}{Situation Calculus}
Para construir sistemas que sean capaces de razonar con información imperfecta, se utiliza una extensión de {\bf Situation Calculus (SC)}, que permite representar el conocimiento de los agentes.\\

\vspace{10pt}
SC es una lógica de predicados creada por McCarthy diseñada para formalizar y automatizar razonamientos y acciones. Tiene elementos predefinidos que permiten:

\begin{itemize}
\item Indicar la situación inicial.
\item Decir que algo es cierto en determianda situación.
\item Decir que una acción es posible en una situación.
\end{itemize}

\end{block}

\begin{block}{¿Cómo razonar a partir de GDL-II?}
Se utiliza una función que mapea elementos de GDL-II en elementos de esta extensión de SC. 
\end{block}


\end{frame}

\begin{frame}
\frametitle{Autores y Trabajos Relacionados}

Este paper fue escrito en 2014 por Stephan Schiffel (Reikjavík University, Islandia) y Michael Thielscher (The University of New South Wales, Australia).\\
Actualmente continúan investigando en el área.\\
\vspace{10pt}
Anualmente la AAAI (Association for the Advancement of Artificial Inteligence) organiza la International General Game Playing Competition, en la que participan equipos de investigación de distintos lugares del mundo.\\
\vspace{10pt}
Se realiza investigación en este área en países como Estados Unidos, Alemania y Australia.
 

\end{frame}


\begin{frame}
\frametitle{References}
\footnotesize{
\begin{thebibliography}{99} % Beamer does not support BibTeX so references must be inserted manually as below
\bibitem[Smith, 2012]{p1} Stephan Schieffel, Michael Thielscher (2014)
\newblock Representing and Reasoning About the Rules of General Games With Imperfect Information
\newblock \emph{Journal of Artificial Intelligence Research} 171 -- 206
\end{thebibliography}
}
\end{frame}
% http://www.cse.unsw.edu.au/~cs4415/2010/resources/stable.pdf

\end{document} 